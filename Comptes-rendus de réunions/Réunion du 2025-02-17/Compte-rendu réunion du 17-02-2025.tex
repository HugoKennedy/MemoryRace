\documentclass[a4paper]{article}

\usepackage[margin=2cm]{geometry}
\usepackage[utf8]{inputenc}
\usepackage[french]{babel}
\renewcommand{\Frlabelitemi}{\textbullet}
\usepackage{fancyhdr}
\usepackage{tikz}\usepackage{pgfplots}
\usepackage{multicol}
\usepackage{wrapfig}
\usepackage{lipsum} % Pour générer du texte fictif
\newcommand{\verticalline}{\vrule width 1.5pt\hspace{5pt}}
\usepackage{amsmath}
\usepackage{amsfonts}
\usepackage{mathtools}
\usepackage{mdframed}
\usepackage{xcolor}
\usepackage{esvect} % Pour les vecteurs
\usepackage{cancel} % pour barrer du texte

%--------------------------------------------------------------------------
% Pour mettre les sections en chiffres romains et les subsections en lettres majuscules

\usepackage{titlesec}
\renewcommand\thesection{\Roman{section}}
\renewcommand{\thesubsection}{\Alph{subsection}}
\renewcommand{\thesubsubsection}{\arabic{subsubsection}}

%--------------------------------------------------------------------------
% Lien cliquable
\usepackage{hyperref}

%--------------------------------------------------------------------------
% Checklist

\usepackage{enumitem,amssymb}
\newlist{todolist}{itemize}{2}
\setlist[todolist]{label=$\square$}
\usepackage{pifont}
\newcommand{\cmark}{\ding{51}}%
\newcommand{\xmark}{\ding{55}}%
\newcommand{\done}{\rlap{$\square$}{\raisebox{2pt}{\large\hspace{1pt}\cmark}}%
	\hspace{-2.5pt}}
\newcommand{\wontfix}{\rlap{$\square$}{\large\hspace{1pt}\xmark}}

%--------------------------------------------------------------------------
% Mise en page
\pagestyle{fancy}
\renewcommand{\footrulewidth}{1pt}
\lhead{MemoryRace}
\chead{}
\rhead{Compte-rendu de réunion}
\lfoot{Alexandre Naizondard / Nicolas Gasca}                      % Nom du rédacteur
\cfoot{\thepage}
\rfoot{Nicolas Nèble / Hugo Kennedy-Martinez}                   % Nom des relecteurs

\begin{document}
	
	\vspace*{-1.2cm} % Ajustez la valeur de l'espace vertical négatif pour rapprocher le cadre du haut de la page
	
	\begin{center}
		\begin{tikzpicture}
			\node[draw,double,thick,inner sep=10pt,double distance=2pt, text width=\textwidth, align=center] {\textbf{\LARGE Compte-rendu de la réunion du 17/02/2025}};
		\end{tikzpicture}
	\end{center}
	
	\section{Présences}
	
	\begin{todolist}
		\item[\done] François Trahay (Tuteur)
		\item[\done] Alexandre Naizondard
		\item[\done] Nicolas Gasca
		\item[\done] Nicolas Nèble
		\item[\done] Hugo Kennedy-Martinez
	\end{todolist}
	
	\section{Ordre du jour}
	
	\begin{itemize}
		\item Fin du livrable 1
	\end{itemize}
	
	\section{Présentation}
	
	\subsection{Fin du livrable 1}
	
	Le livrable 1 a été terminé et déposé sur gitlab. Observation rapide du livrable, validation des éléments par M. Trahay. Il est donc temps d'attaquer la partie plus concrète.
	
	M. Trahay nous a demandé si nous avons commencé à coder ou à se former.
	\begin{itemize}
		\item Nicolas N : vu les bases de données avec Kaggle
		\item Nicolas G : on se demandait comment on allait stocker les images, vu des solutions comme redis. La difficulté est que plutôt qu’ouvrir avec un script python, il faut configurer l’interfaçage avec Redis.
	\end{itemize}
	
	M. Trahay nous a demandé si nous avons déjà fait des interfaces graphiques en Java. Hugo répond que non mais qu'il y a du bon contenu sur YouTube à ce sujet, notamment Bro code qui a fait une longue formation Java et aborde aussi les interfaces graphiques.
	M. Trahay explique alors que JavaFX n'est pas génial lorsqu'il s'agit de créer des boutons, etc. C'est surtout bien pour les fenêtres de jeu vidéo. Pour les applications, il existe d’autres bibliothèques telles que Swing. A nous de regarder les bibliothèques qui existent pour ça. Ça reste de la programmation événementielle mais plus d’outils pour s’occuper de cela.
	
	\section{Remarques et conseils de M. Trahay}
	
	\begin{itemize}
		\item Regarder les différentes interfaces graphiques qui existent
		\item Une fois la base de données mise en place, commencer à voir comment on fait le lien
		\item Pour la base de donnée, peut être la faire en local, c'est plus simple. Mais c'est aussi bien d’avoir une base de données centralisée.
	\end{itemize}
	
	\section{A faire pour la prochaine fois}
	
	\begin{itemize}
		\item Choix de l'interface graphique
		\item Première version d'un diagramme de classe UML
		\item Première version de l'algorithme d'extraction
	\end{itemize}
	
	\centering\Huge\textbf{Prochaine réunion :}\\
	\centering\Huge\textbf{Mardi 11 Mars à 15h, D306}
	
\end{document}