\documentclass[a4paper]{article}

\usepackage[margin=2cm]{geometry}
\usepackage[utf8]{inputenc}
\usepackage[french]{babel}
\renewcommand{\Frlabelitemi}{\textbullet}
\usepackage{fancyhdr}
\usepackage{tikz}\usepackage{pgfplots}
\usepackage{multicol}
\usepackage{wrapfig}
\usepackage{lipsum} % Pour générer du texte fictif
\newcommand{\verticalline}{\vrule width 1.5pt\hspace{5pt}}
\usepackage{amsmath}
\usepackage{amsfonts}
\usepackage{mathtools}
\usepackage{mdframed}
\usepackage{xcolor}
\usepackage{esvect} % Pour les vecteurs
\usepackage{cancel} % pour barrer du texte

%--------------------------------------------------------------------------
% Pour mettre les sections en chiffres romains et les subsections en lettres majuscules

\usepackage{titlesec}
\renewcommand\thesection{\Roman{section}}
\renewcommand{\thesubsection}{\Alph{subsection}}
\renewcommand{\thesubsubsection}{\arabic{subsubsection}}

%--------------------------------------------------------------------------
% Lien cliquable
\usepackage{hyperref}

%--------------------------------------------------------------------------
% Checklist

\usepackage{enumitem,amssymb}
\newlist{todolist}{itemize}{2}
\setlist[todolist]{label=$\square$}
\usepackage{pifont}
\newcommand{\cmark}{\ding{51}}%
\newcommand{\xmark}{\ding{55}}%
\newcommand{\done}{\rlap{$\square$}{\raisebox{2pt}{\large\hspace{1pt}\cmark}}%
	\hspace{-2.5pt}}
\newcommand{\wontfix}{\rlap{$\square$}{\large\hspace{1pt}\xmark}}

%--------------------------------------------------------------------------
% Mise en page
\pagestyle{fancy}
\renewcommand{\footrulewidth}{1pt}
\lhead{MemoryRace}
\chead{}
\rhead{Compte-rendu de réunion}
\lfoot{Alexandre Naizondard / Nicolas Gasca}                      % Nom du rédacteur
\cfoot{\thepage}
\rfoot{Nicolas Nèble / Hugo Kennedy-Martinez}                   % Nom des relecteurs

\begin{document}
	
	\vspace*{-1.2cm} % Ajustez la valeur de l'espace vertical négatif pour rapprocher le cadre du haut de la page
	
	\begin{center}
		\begin{tikzpicture}
			\node[draw,double,thick,inner sep=10pt,double distance=2pt, text width=\textwidth, align=center] {\textbf{\LARGE Compte-rendu de la réunion du 05/05/2025}};
		\end{tikzpicture}
	\end{center}
	
	\section{Présences}
	
	\begin{todolist}
		\item[\done] François Trahay (Tuteur)
		\item[\done] Alexandre Naizondard
		\item[\done] Nicolas Gasca
		\item[\done] Nicolas Nèble
		\item[\done] Hugo Kennedy-Martinez
	\end{todolist}
	
	\section{Ordre du jour}
	
	À signaler : la réunion s'est faite très rapidement pour des contraintes de temps.
	
	\begin{itemize}
		\item Avancés sur le GUI
		\item Avancés sur les requêtes
	\end{itemize}
	
	\section{Présentation}
	
	\subsection{Avancés sur le GUI}
	
	Nicolas Nèble dit que l’interface graphique a pas mal évoluée. On a : 
	\begin{itemize}
		\item Séparation entre coureur et organisateur
		\item Pour le coureur : sélectionner la  course et le numéro de dossard et ça récupère les photos.
		\item Pour l'organisateur : pour la création de course, cela demande le nom de la course, la date et le lieu pour les envoyer à la base de donnée ; et on a aussi la possibilité de modifier la course (nom, lieu, ajouter de nouvelles photos)
	\end{itemize}
	Il fait également remarquer que pour l'instant, les photos s'affichage individuellement par fenêtre et qu'il serait mieux de faire une seule fenêtre de type galerie.
	
	\subsection{Avancés sur les requêtes}

	On a une sépraration entre les requêtes et l’interface grapgique. La structure actuelle permet une plus grande abstraction. Il ne reste que quelques requêtes à terminer.
	
	M. Trahay demande si l’upload des photos lance le script de reconnaissance ? Alexandre répond que l'idée actuelle est que l'on ajoute les photos dans la base de données puis ça lance le script.
	
	\section{Remarques et conseils de M. Trahay}
	
	\begin{itemize}
		\item La soutenance ne sera en fait pas sur les créneaux indiqués car il est en déplacement toute cette semaine (semaine du 2 juin), donc cela sera sûrement la semaine suivante. Il doit juste trouver une date et un président de jury.
	\end{itemize}
	
	\section{A faire pour la prochaine fois}
	
	\begin{itemize}
		\item côté affichage de la galerie.
		\item fin des requêtes à faire.
		\item Lien Database et script python.
		\item En résumé, terminer le tout pour le rendu de la version finale le 17 mai.
	\end{itemize}
	
	\centering\Huge\textbf{Pas de prochaine réunion prévue}\\
	
\end{document}